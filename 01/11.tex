\subsection{%
  Лекция \texttt{24.04.18}.%
}

\subheader{Математическое ожидание и дисперсия случайного вектора}

Пусть имеется случайный вектор \(\vec{X} = \prh{X_1, \dotsc, X_n}^T\), где
случайные величины \(X_i\) это компоненты случайного вектора.

\begin{definition}
  Математическим ожиданием случайного вектора \(\vec{X}\) называется вектор
  с координатами из математических ожиданий компонент.

  \begin{equation*}
    \expected{\vec{X}} = \begin{pmatrix}
      \expected{X_1} \\
      \vdots \\
      \expected{X_n}
    \end{pmatrix}
  \end{equation*}
\end{definition}

\begin{definition}
  Дисперсией (или матрицей ковариаций) случайного вектора \(\vec{X}\) называется
  матрица

  \begin{equation*}
    \variance{\vec{X}}
    = \expected{\vec{X} - \expected{X}} \prh{\vec{X} - \expected{X}}^T
  \end{equation*}

  Или, более просто, это матрица, состоящая из элементов \(d_{i, j} =
  \cov{X_i}{X_j}\).
\end{definition}

\begin{remark}
  На главной диагонали матрицы ковариаций стоят дисперсии компонент.
\end{remark}

\subsubheader{}{Свойства}

\begin{enumerate}
\item
  \(\expected{A \vec{X}} = A \expected{\vec{X}}\), где \(A\) это матрица размера
  \(n \times n\).

\item
  \(\expected{\vec{X} + \vec{B}} = \expected{\vec{X}} + \vec{B}\)

\item
  \(\variance{A \vec{X}} = A \variance{\vec{X}} A^T\)

\item
  \(\variance{\vec{X} + \vec{B}} = \variance{\vec{X}}\)
\end{enumerate}

\subheader{Общая модель линейной регрессии}

Пусть результат \(X\) зависит от факторов \(Z_1, \dotsc, Z_k\). Рассматривается
теоретическая модель линейной регрессии.

\begin{equation*}
  \expected{\vec{X} \given \vec{Z}}
  = f \prh{\vec{Z}}
  = \beta_1 Z_1 + \dotsc + \beta_k Z_k + \epsilon
\end{equation*}

где \(\vec{Z} = \prh{z_1, \dotsc, z_k}^T\). Обозначим \(\vec{\beta} =
\prh{\beta_1, \dotsc, \beta_k}^T\)~--- вектор неизвестных параметров регрессии,
\(\epsilon\)~--- случайный член (ошибка), отражающая нелинейность модели,
влияние неучтенных факторов и т.д.

Пусть проведено \(n \ge k\) экспериментов. Обозначим \(\vec{Z}^{(i)} =
\prh{z_1^{(i)}, \dotsc, z_k^{(i)}}\) набор значений факторов при \(i\)-ом
эксперименте. Пусть \(\vec{X} = \prh{X_1, \dotsc, X_n}\) это набор результатов
при \(n\) экспериментах. Согласно модели

\begin{equation*}
  \begin{cases}
    X_1 = \beta_1 z_1^{(1)} + \dotsc + \beta_k z_k^{(1)} + \epsilon_1 \\
    X_2 = \beta_1 z_1^{(2)} + \dotsc + \beta_k z_k^{(2)} + \epsilon_2 \\
    \vdots \\
    X_n = \beta_1 z_1^{(n)} + \dotsc + \beta_k z_k^{(n)} + \epsilon_n
  \end{cases}
\end{equation*}

где \(\epsilon_i\) это теоретическая ошибка при \(i\)-ом эксперименте (она
неизвестна). В матричной форме

\begin{equation*}
  \vec{X} = Z^T \vec{\beta} + \vec{\epsilon}
\end{equation*}

где \(\vec{\epsilon} = \prh{\epsilon_1, \dotsc, \epsilon_n}\) это столбец
ошибок, а матрица \(Z\) называется матрицей плана и имеет вид

\begin{equation*}
  Z_{k \times n} = \begin{pmatrix}
    z_1^{(1)} & z_1^{(2)} & \dotsc & z_1^{(n)} \\
    z_2^{(1)} & z_2^{(2)} & \dotsc & z_2^{(n)} \\
    \vdots    & \vdots    & \ddots & \vdots    \\
    z_k^{(1)} & z_k^{(2)} & \dotsc & z_k^{(n)} \\
  \end{pmatrix}
\end{equation*}

Требуется по данной матрице плана \(Z\) и вектору результатов \(\vec{X}\) найти
оценки \(\vec{B} = \prh{b_1, \dotsc, b_k}\) для параметров регрессии
\(\vec{\beta} = \prh{\beta_1, \dotsc, \beta_k}\) и параметров ошибок
\(\epsilon_i\).

\begin{remark}
  В данной модели мы не теряем свободный член \(a\), т.к. можно считать, что
  \(z_1 \equiv 1\) и ей соответствует строка \(\prh{1, \dotsc, 1}\) в матрице
  плана \(Z\).
\end{remark}

\subheader{Метод наименьших квадратов и нормальные уравнения}

Будем считать, что выполнено условие о том, что \(\rank Z = k\), т.е. все строки
матрицы плана независимы. Обозначим \(A = Z Z^T\)~--- это будет квадратная
матрица размера \(k \times k\). Заметим, что

\begin{enumerate}
\item
  Матрица \(A\) симметричная, т.е. \(A^T = A\).

\item
  Матрица \(A\) будет положительно определенной.

\item
  Существенная вещественная матрица \(\sqrt{A}\) такая, что \(\prh{\sqrt{A}}^2 =
  A\).
\end{enumerate}

Экспериментальная модель имеет вид

\begin{equation*}
  X_i = b_1 Z_1 + \dotsc b_k Z_k + \widehat{\epsilon}_i
\end{equation*}

где \(\widehat{\epsilon}_i = X_i - \prh{b_1 Z_1 + \dotsc + b_k Z_k}\).

При методе наименьших квадратов находим оценку \(\vec{B} = \prh{b_1, \dotsc,
b_k}^T\), которая минимизирует функцию

\begin{equation*}
  L \prh{\vec{B}}
  = \sum_{i = 1}^n \widehat{\epsilon}_i^2
  = \norm{\widehat{\epsilon}}^2
  = \norm{\vec{X} - Z^T \vec{B}}^2
\end{equation*}

Заметим, что \(\norm{\vec{X} - Z^T \vec{B}}^2\) это квадрат расстояния от точки
\(\vec{X} \in \RR^n\) до точки \(Z^T \vec{B} \in \RR^n\). Причем \(Z^T \vec{B}\)
это произвольная точка подпространства \(Z^T \vec{t}\), где \(\vec{t} \in
\RR^k\). Заметим, что \(\dim Z^T \vec{t} = k\), т.к. все строки матрицы плана
независимы. Таким образом, искомое минимальное расстояние это расстояние от
точки \(\vec{X} \in \RR^n\) до подпространства \(Z^T \vec{t}, \vec{t} \in
\RR^k\). Это длина перпендикуляра, т.е. нужное расстояние получаем при условии,
что вектор \(\vec{X} - Z^T \vec{B}\) будет ортогонален всем векторам данного
подпространства. Значит \(\forall \vec{t} \in \RR^k\) справедливо

\begin{equation*}
  \dotpdt{Z^T \vec{t}}{\vec{X} - Z^T \vec{B}}
  = \prh{Z^T \vec{t}}^T \cdot \prh{\vec{X} - Z^T \vec{B}}
  = \vec{t}^T Z \prh{\vec{X} - Z^T \vec{B}}
  = \vec{t}^T \prh{Z \vec{X} - Z Z^T \vec{B}}
  = \vec{t}^T \prh{Z \vec{X} - A \vec{B}}
  = 0
\end{equation*}

Т.к. всем векторам данного пространства может быть ортогонален только нулевой
вектор, то получаем, что

\begin{equation*}
  Z \vec{X} - A \vec{B} = \vec{0}
  \iff
  A \vec{B} = Z \vec{X}
  \iff
  \vec{B} = A^{-1} Z \vec{X}
\end{equation*}

Это нормальное уравнение или система нормальных уравнений с неизвестными \(b_1,
\dotsc, b_k\). Получили оценки по методу наименьших квадратов.

\subheader{Свойства оценок по методу наименьших квадратов}

Далее предполагаем, что выполнено условие \(\rank Z = k\) и ошибки
\(\epsilon_i\)~--- независимые нормальные случайные величины с распределением
\(\ndist{0}{\sigma^2}\). Таким образом \(\variance{\vec{\epsilon}} = \sigma^2
E_n\).

\begin{lemma}
  \begin{equation*}
    \vec{B} - \vec{\beta} = A^{-1} Z \vec{\epsilon}
  \end{equation*}  
\end{lemma}

\begin{proof}
  \begin{equation*}
    \begin{aligned}
      \vec{B} - \vec{\beta}
      & = A^{-1} Z \vec{X} - \vec{\beta}
    \\
      & = A^{-1} Z \prh{Z^T \vec{\beta} + \vec{\epsilon}} - \vec{\beta}
    \\
      & = A^{-1} Z Z^T \vec{\beta} + A^{-1} Z \vec{\epsilon} - \vec{\beta}
    \\
      & = A^{-1} A \vec{\beta} + A^{-1} Z \vec{\epsilon} - \vec{\beta}
    \\
      & = \vec{\beta} + A^{-1} Z \vec{\epsilon} - \vec{\beta}
    \\
      & = A^{-1} Z \vec{\epsilon}
    \end{aligned}
  \end{equation*}
\end{proof}

\begin{lemma}
  Оценка \(\vec{B}\) это несмещенная оценка для параметра \(\vec{\beta}\).
\end{lemma}

\begin{proof}
  \begin{equation*}
    \expected{\vec{B} - \vec{\beta}}
    = \expected{A^{-1} Z \vec{\epsilon}}
    = A^{-1} Z \expected{\vec{\epsilon}}
    = A^{-1} Z \vec{0}
    = \vec{0}
    \implies
    \expected{\vec{B}} = \vec{\beta}
  \end{equation*}  
\end{proof}

\begin{lemma}
  \begin{equation*}
    \variance{\vec{B}} = \sigma^2 A^{-1}
  \end{equation*}
\end{lemma}

\begin{proof}
  \begin{equation*}
    \begin{aligned}
      \variance{\vec{B}}
      & = \variance{\vec{B} - \vec{\beta}}
    \\
      & = \variance{A^{-1} Z \vec{\epsilon}}
    \\
      & = A^{-1} Z \variance{\vec{\epsilon}} \prh{A^{-1} Z}^T
    \\
      & = A^{-1} Z \sigma^2 E_n Z^T A^{-1}
    \\
      & = \sigma^2 A^{-1} Z Z^T A^{-1}
    \\
      & = \sigma^2 A^{-1} A A^{-1}
    \\
      & = \sigma^2 A^{-1}
    \end{aligned}
  \end{equation*}
\end{proof}

\begin{remark} \label{rem:variance-b-i}
  Таким образом \(\display{\variance{b_i} = \sigma^2 \prh{A^{-1}}_{i, i}}\).
\end{remark}

Введем обозначение \(\widehat{\sigma}^2\), которое определим как

\begin{equation*}
  \widehat{\sigma}^2
  = \frac{1}{n} \sum_{i = 1}^n \widehat{\epsilon}_i^2
  = \frac{1}{n} \norm{\vec{X} - Z^T \vec{B}}
  = \frac{1}{n} L \prh{\vec{B}}
\end{equation*}

Заметим, что \(\widehat{\sigma}^2\) это оценка неизвестной дисперсии ошибки
\(\sigma^2\).

\begin{theorem} \label{thr:md-err}
  Пусть выполнено условие \(\rank Z = k\) и ошибки \(\epsilon_i\)~---
  независимые нормальные случайные величины с распределением
  \(\ndist{0}{\sigma^2}\). Тогда

  \begin{enumerate}
  \item
    Вектор \(\display{\frac{1}{\sigma} \sqrt{A} \prh{\vec{B} - \vec{\beta}}}\)
    состоит из независимых случайных величин со стандартным нормальным
    распределением.

  \item
    \begin{equation*}
      \frac{n \widehat{\sigma}^2}{\sigma^2} \in \Hdist{n - k}
      \text{ и не зависит от } \vec{B}
    \end{equation*}

  \item
    \begin{equation*}
      S^2
      = \frac{n \widehat{\sigma}^2}{(n - k) \sigma^2}
      = \frac{1}{n - k} \sum_{i = 1}^n \widehat{\epsilon}_i^2 
      \qquad
      \text{Это несмещенная оценка для } \sigma^2
    \end{equation*}
  \end{enumerate}
\end{theorem}

\begin{proof}
  \subsubheader{I}{ пункт}

  \begin{equation*}
    \frac{1}{\sigma} \sqrt{A} \prh{\vec{B} - \vec{\beta}}
    = \frac{1}{\sigma} \sqrt{A} A^{-1} Z \vec{\epsilon}
    = \frac{1}{\sigma} \sqrt{A^{-1}} Z \vec{\epsilon}
  \end{equation*}

  Получили линейное преобразование нормального вектора \(\vec{\epsilon}\), и
  поэтому оно также является нормальным вектором.

  \begin{equation*}
    \expected{\frac{1}{\sigma} \sqrt{A} \prh{\vec{B} - \vec{\beta}}}
    = \frac{1}{\sigma} \sqrt{A} \expected{\vec{B} - \vec{\beta}}
    = \frac{1}{\sigma} \sqrt{A} \vec{0}
    = \vec{0}
  \end{equation*}

  Таким образом первый параметр компонент равен нулю.

  \begin{equation*}
    \variance{\frac{1}{\sigma} \sqrt{A} \prh{\vec{B} - \vec{\beta}}}
    = \frac{1}{\sigma^2} \sqrt{A} \variance{\vec{B} - \vec{\beta}} \sqrt{A^T}
    = \frac{1}{\sigma^2} \sqrt{A} \sigma^2 A^{-1} \sqrt{A}
    = E_k
  \end{equation*}

  Значит все координаты не коррелированны и имеют дисперсию равную единице.

  \subsubheader{II}{ пункт}

  По построению метода наименьших квадратов имеем

  \begin{equation*}
    \forall \vec{t} \in \RR^k \given \vec{X} - Z^T \vec{B} \perp Z^T \vec{t}
  \end{equation*}

  В частости это верно для \(\vec{t} = \vec{B} - \vec{\beta}\). По обобщенной
  теореме Пифагора

  \begin{equation*} \label{eq:md-err-1} \tag{1}
    \begin{aligned}
      \norm{\vec{X} - Z^T \vec{B}}^2 + \norm{Z^T \prh{\vec{B} - \vec{\beta}}}^2
      = \norm{\vec{X} - Z^T \vec{B} + Z^T \prh{\vec{B} - \vec{\beta}}}^2
      = \norm{\vec{X} - Z^T \vec{\beta}}^2
    \\
      \norm{\vec{X} - Z^T \vec{B}}^2
      = \norm{\vec{X} - Z^T \vec{\beta}}^2
        - \norm{Z^T \prh{\vec{B} - \vec{\beta}}}^2
      = \norm{\vec{\epsilon}}^2 - \norm{Z^T \prh{\vec{B} - \vec{\beta}}}^2
    \end{aligned}
  \end{equation*}

  Далее работает со вторым слагаемым.

  \begin{equation*} \label{eq:md-err-2} \tag{2}
    \begin{aligned}
      \norm{Z^T \prh{\vec{B} - \vec{\beta}}}^2
      & = \prh{\vec{B} - \vec{\beta}}^T Z Z^T \prh{\vec{B} - \vec{\beta}}
    \\
      & = \prh{\vec{B} - \vec{\beta}}^T A \prh{\vec{B} - \vec{\beta}}
    \\
      & = \prh{\vec{B} - \vec{\beta}}^T \sqrt{A^T}
        \sqrt{A} \prh{\vec{B} - \vec{\beta}}
    \\
      & = \norm{\sqrt{A} \prh{\vec{B} - \vec{\beta}}}^2
    \\
      & = \norm{\sqrt{A} A^{-1} Z \vec{\epsilon}}^2
    \\
      & = \norm{\sqrt{A^{-1}} Z \vec{\epsilon}}^2
    \end{aligned}
  \end{equation*}

  Заметим, что строки матрицы \(\sqrt{A^{-1}} Z\) ортогональны, т.к.

  \begin{equation*}
    \prh{\sqrt{A^{-1}} Z} \prh{\sqrt{A^{-1}} Z}^T
    = \sqrt{A^{-1}} Z Z^T \prh{\sqrt{A^{-1}}}^T
    = \sqrt{A^{-1}} A \sqrt{A^{-1}}
    = E
  \end{equation*}

  \(\sqrt{A^{-1}} Z\) это прямоугольная матрица размера \(k \times n\). Из курса
  линейной алгебры известно, что ее можно дополнить до ортогональной матрицы
  \(C\) размера \(n \times n\). Тогда первые \(k\) координат \(n\)-мерного
  вектора \(\vec{Y} = \frac{1}{\sigma} C \vec{\epsilon}\) совпадают с
  координатами вектора \(\vec{Y} = \frac{1}{\sigma} \sqrt{A^{-1}} Z
  \vec{\epsilon}\).

  В результате из \eqref{eq:md-err-1} и \eqref{eq:md-err-2} получаем

  \begin{equation*} \label{eq:md-err-3} \tag{3}
    \begin{aligned}
      \frac{n \widehat{\sigma}^2}{\sigma^2}
      & = \frac{1}{\sigma^2} \sum_{i = 1}^n \widehat{\epsilon}_i^2
    \\
      & = \frac{1}{\sigma^2} \norm{\vec{X} - Z^T \vec{B}}^2
    \\
      & \eqby{\eqref{eq:md-err-1}}
        \frac{1}{\sigma^2} \norm{\vec{\epsilon}}^2
          - \frac{1}{\sigma^2} \norm{Z^T \prh{\vec{B} - \vec{\beta}}}^2
    \\
      & = \norm{\frac{\vec{\epsilon}}{\sigma}}^2
          - \norm{\frac{Z^T \prh{\vec{B} - \vec{\beta}}}{\sigma}}^2
    \\
      & \eqby{\eqref{eq:md-err-2}}
        \norm{\frac{\vec{\epsilon}}{\sigma}}^2
          - \norm{\frac{\sqrt{A^{-1}} Z \vec{\epsilon}}{\sigma}}^2
    \\
      & = \sum_{i = 1}^n \prh{\frac{\epsilon_i}{\sigma}}^2
        - Y_1^2 - Y_2^2 - \dotsc - Y_k^2
    \end{aligned}
  \end{equation*}

  Вектор \(\frac{\vec{\epsilon}}{\sigma}\) имеет \(n\)-мерное стандартное
  нормальное распределение, а \(Y_1, \dotsc, Y_k\) это первые \(k\) координат
  ортогонального преобразования данного вектора, поэтому \eqref{eq:md-err-3}
  согласно \ref{lem:fisher} имеет распределение \(\Hdist{n - k}\) и не зависит
  от вычитаемых координат вектора \(\frac{\sqrt{A^{-1}} Z
  \vec{\epsilon}}{\sigma}\), а значит и от вектора

  \begin{equation*}
    \vec{B}
    = A^{-1} Z \vec{\epsilon} + \vec{\beta}
    = \sigma \sqrt{A^{-1}} \prh{\frac{1}{\sigma} \sqrt{A^{-1}} Z \vec{\epsilon}}
  \end{equation*}

  которые являются их линейными комбинациями.

  \subsubheader{III}{ пункт}

  Т.к. \(\expected{\chi_{n - k}^2} = n - k\), то

  \begin{equation*}
    \expected{\widehat{\sigma}^2}
    = \frac{\sigma^2}{n} \expected{\frac{n \widehat{\sigma}^2}{\sigma^2}}
    = \frac{\sigma^2}{n} \cdot \prh{n - k}
    = \frac{n - k}{n} \cdot \sigma^2
  \end{equation*}

  Это не равно \(\sigma^2\), т.е. оценка будет смещенной. Значит оценка

  \begin{equation*}
    S^2
    = \frac{n}{n - k} \widehat{\sigma}^2
    = \frac{1}{n - k} \sum_{i = 1}^n \widehat{\epsilon}_i^2
  \end{equation*}

  будет несмещенной.
\end{proof}
