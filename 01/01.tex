\subsection{%
  Лекция \texttt{24.02.08}.%
}

\subheader{Выборки}

\begin{definition}
  Под генеральной совокупностью понимают все результаты данной серии
  экспериментов (или экспериментальных значений случайной величины).
\end{definition}

\begin{definition}
  Под выборочной совокупностью понимают имеющиеся у нас данные (выборка из
  генеральной совокупности, возможно неполная).
\end{definition}

\begin{remark}
  Выборочная совокупность не всегда отражает реальное поведение случайной
  величины. В качестве иллюстрации можно рассматривать известный пример с
  бронированием самолетов (классическая ошибка выжившего).
\end{remark}

\begin{definition}
  Репрезентативной выборкой называется выборка, имеющая то же самое
  распределение, что и у генеральной совокупности.
\end{definition}

\begin{remark}
  В дальнейшем в курсе предполагаем, что все выборки репрезентативные.
\end{remark}

\begin{definition}[Первое]
  Выборкой объема \(n\) называется набор экспериментальных данных \((x_1,
  \dotsc, x_n)\).
\end{definition}

\begin{definition}[Второе]
  Выборкой объема \(n\) называется набор \(X_1, \dotsc, X_n\) независимых
  одинаково распределенных случайных величин.
\end{definition}

\begin{remark}
  Таким образом \(\expected{X_i} = \expected{X_1}\) и \(\variance{X_i} =
  \variance{X_1}\), поэтому обычно числовые характеристики \(i\)-того заменяются
  числовыми характеристиками первого экземпляра.
\end{remark}

\subheader{Выборочные характеристики}

Выборку можно рассматривать как дискретную случайную величину.

\begin{ttable}{0.4 \linewidth}{X|X|X|X}
  \(X\)   & \(x_1\)         & \(\dotsc\) & \(x_n\)
  \\ \hline
  \(p^*\) & \(\frac{1}{n}\) & \(\dotsc\) & \(\frac{1}{n}\)
\end{ttable}

\begin{definition}
  Средним выборочным \(\avg{x}\) называется число

  \begin{equation*}
    \avg{x}
    = \frac{x_1 + \dotsc + x_n}{n}
    = \frac{1}{n} \sum_{i = 1}^n x_i
  \end{equation*}

  Это будет оценкой неизвестного математического ожидания.
\end{definition}

\begin{definition}
  Выборочной дисперсией \(\svarianceD\) называется число

  \begin{equation*}
    \svarianceD = \frac{1}{n} \sum_{i = 1}^n \prh{x_i - \avg{x}}^2
  \end{equation*}

  Это будет оценкой неизвестной дисперсии.
\end{definition}

\begin{definition}
  Выборочным среднеквадратическим отклонение называется число

  \begin{equation*}
    \sstderS = \sqrt{\svarianceD}
  \end{equation*}

  Это будет оценкой неизвестного среднеквадратического отклонения.
\end{definition}

\begin{definition}
  Выборочной функцией распределения \(F^* (y)\) называется функция

  \begin{equation*}
    F^* (y) = \frac{\text{число данных } x_i \in \interval{-\infty}{y}}{n}
  \end{equation*}
\end{definition}

\begin{theorem}[Гливенко---Кантелли]
  Пусть имеется выборка \(\sample{X} = (x_1, \dotsc, x_n)\) объема \(n\).
  Обозначим \(F^* (y)\)~--- эмпирическую функцию распределения, а \(F(y)\)~---
  теоретическую функцию распределения. Тогда

  \begin{equation*}
    \sup_{y \in \RR} \abs{F^* (y) - F(y)} \Rarr{\probP} 0
    \qquad
    \text{при } n \to \infty
  \end{equation*}
\end{theorem}

\subheader{Начальная обработка статданных}

\subsubheader{I.}{Ранжирование выборки}

Упорядочиваем данные по возрастанию, в результате получаем вариационный ряд вида
\(X_{(1)}, \dotsc, X_{(n)}\).

\begin{definition}
  Разность \(X_{(n)} - X_{(1)}\) называется размахом выборки.
\end{definition}

\begin{definition}
  Элемент \(X_{(i)}\) в полученном ряде называется \(i\)-той порядковой
  статистикой.
\end{definition}

\begin{remark}
  Если мы при этом объединяем повторяющиеся результаты (с учетом числа этих
  результатов), то получаем частотный вариационный ряд.
\end{remark}

\subsubheader{II.}{Разбиение на интервалы}

Если много неповторяющихся данных, то разбиваем их на интервалы и составляем
интервальный вариационный ряд.

\begin{remark}
  Есть два подхода к разбиению на интервалы.

  \begin{enumerate}
  \item
    Берем интервалы одинаковой длины. Это удобнее для построения гистограммы и
    выдвижения гипотезы и типе распределения.

  \item
    Берем равнонаполненные интервалы. Это удобнее при проверке гипотез о типе
    распределения.
  \end{enumerate}
\end{remark}

\begin{remark}
  Обычно (но далеко не всегда) количество интервалов вычисляется по формуле
  Стерджеса.

  \begin{equation*}
    K \approx 1 + \log_2 n
  \end{equation*}
\end{remark}

В результате получаем \(K\) интервалов \([a_{i - 1}; a_i)\) и считаем их
частоты \(v_i\)~--- число данных, попавших в \(i\)-ый интервал. Величина
\(\frac{v_i}{n}\) называется относительной частотой (она является оценкой
теоретической вероятности попадания случайной величины в данный интервал). Если
заменяем интервалы на их середины \(c_i = \frac{a_{i - 1} + a_i}{2}\), то опять
получаем огрубленный вариационный ряд, по которому можно дать точные оценки
числовым характеристикам распределения. Для этого можно использовать формулы

\begin{equation*}
  \avg{x} = \frac{1}{n} \sum_{i = 1}^n c_i v_i
  \qquad
  \svarianceD = \frac{1}{n} \sum_{i = 1}^n \prh{c_i - \avg{x}}^2 v_i
\end{equation*}

\subheader{Геометрическая интерпретация данных}

Обычно удобнее визуализировать данные в виде гистограммы. На плоскости строим
набор прямоугольников для каждого интервала. Основание это \([a_{i - 1}; a_i)\)
длины \(l = a_i - a_{i - 1}\), а высоту берем пропорционально частоте, причем
таким образом, чтобы общая площадь фигуры равнялась единице, т.е. высота будет
равна \(\frac{v_i}{n l}\). Гистограмма является приближением плотности
распределения, и по ее виду можно выдвинуть гипотезы о типе распределения.

\begin{theorem}
  Если число интервалов \(K(n) \to \infty\) и \(\frac{K(n)}{n} \to 0\) при \(n
  \to \infty\), то гистограмма по вероятности поточечно сходится к теоретической
  плотности.
\end{theorem}
