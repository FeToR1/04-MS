\subsection{%
  Лекция \texttt{24.03.07}.%
}

\subheader{Квантили распределений}

Предполагаем, что распределение абсолютно непрерывно и \(F(X)\)~--- его функция
распределения.

\begin{definition}[A]
  Число \(t_{\gamma}\) называется квантилем распределения уровня \(\gamma\),
  если \(F \prh{t_{\gamma}} = \gamma\). Таким образом \(t_{\gamma} = F^{-1}
  (\gamma)\).
\end{definition}

\begin{remark}
  Медиана это квантиль уровня \(\frac{1}{2}\).
\end{remark}

\todo image 4:30

В ряде источников под квантилем понимается несколько иное.

\begin{definition}[B]
  Число \(t_{\alpha}\) называется квантилем уровня значимости \(\alpha\), если
  \(\prob{x > t_{\alpha}} = \alpha\) или \(F \prh{t_{\alpha}} = 1 - \alpha\).
  Таким образом \(t_{\alpha}\) соответствует вероятности попадания величины в
  правостороннюю область.
\end{definition}

\begin{remark}
  Будем использовать оба определения, но во втором случае добавлять
  \quote{уровня значимости} и использовать символ \(\alpha\), а не \(\gamma\).
\end{remark}

\subsubheader{Квантили основных распределений в Excel}

\begin{enumerate}
\item
  \texttt{НОРМ.СТ.ОБР}~--- обратная функция к функции стандартного нормального
  распределения.

  \begin{equation*}
    F(x)
    = \frac{1}{\sqrt{2 \pi}} \int_{-\infty}^x \exp \prh{-\frac{z^2}{2}} \dd z
  \end{equation*}

  Тогда \texttt{НОРМ.СТ.ОБР \((x + 0.5)\)} это обратная функция к функции
  Лапласа \(\Phi (x)\).

  \begin{equation*}
    \Phi (x)
    = \frac{1}{\sqrt{2 \pi}} \int_0^x \exp \prh{-\frac{z^2}{2}} \dd z
  \end{equation*}

\item
  \begin{enumerate}
  \item
    \texttt{СТЬЮДЕНТ.ОБР \((\gamma, n)\)}~--- функция, обратная к функции
    распределения Стьюдента с \(n\) степенями свободы.

    \begin{equation*}
      t_n = \frac{X_0}{\sqrt{\frac{1}{n} \chi_n^2}}
    \end{equation*}

    Заметим, что это распределение симметричное относительно нулю, поэтому можно
    использовать другой способ.

  \item
    \texttt{СТЬЮДЕНТ.ОБР.2Х}~--- так называемое двустороннее обратное
    преобразование Стьюдента, которое возвращает \(t_{\alpha}\) такое, что
    \(\prob{\abs{X} > t_{\alpha}} = \alpha\).

    Таким образом \(\prob{\abs{X} < t_{\alpha}} = 1 - \alpha\), значит
    \(t_{\gamma} = \) \texttt{СТЬЮДЕНТ.ОБР.2Х \((1 - \gamma, n)\)}.
  \end{enumerate}

\item
  \begin{enumerate}
  \item
    \texttt{ХИ2.ОБР}~--- возвращает квантиль \(t_{\gamma}\) для распределения
    \(\Hdist{n}\) с \(n\) степенями свободы.

  \item
    \texttt{ХИ2.ОБР.ПХ}~--- возвращает квантиль \(t_{\alpha}\). Заметим, что
    \texttt{ХИ2.ОБР \((\gamma, n) = \) ХИ2.ОБР.ПХ \((1 - \gamma, n)\)}.
  \end{enumerate}

\item
  \begin{enumerate}
  \item
    \texttt{F.ОБР}~--- возвращает \(t_{\gamma}\) для \(F\)-распределения.

  \item
    \texttt{F.ОБР.ПХ}~--- возвращает \(t_{\alpha}\). Заметим, что \texttt{F.ОБР
    \((\gamma, n, m) = \) F.ОБР.ПХ \((1 - \gamma, n, m)\)}.
  \end{enumerate}
\end{enumerate}

\subheader{Интервальные оценки}

\begin{definition}
  Интервал \(\interval{\Theta_{\gamma}^-}{\Theta_{\gamma}^+}\) называется
  доверительным интервалом для параметра \(\Theta\) уровня надежности \(\gamma\)
  если

  \begin{equation*}
    \Theta \in \interval{\Theta_{\gamma}^-}{\Theta_{\gamma}^+}
    \text{ с вероятностью } \gamma
    \qquad
    \text{ или }
    \qquad
    \prob{\Theta_{\gamma}^- < \Theta < \Theta_{\gamma}^+} = \gamma
  \end{equation*}
\end{definition}

\begin{remark}
  В случае дискретного распределения более точным будет определение

  \begin{equation*}
    \prob{\Theta_{\gamma}^- < \Theta < \Theta_{\gamma}^+} \ge \gamma
  \end{equation*}
\end{remark}

\begin{remark}
  Неверно говорить, что \(\Theta\) лежит в данном интервале с вероятностью
  \(\gamma\), т.к. параметр \(\Theta\) не является случайной величиной.
  Правильнее говорить, что доверительный интервал накрывает параметр \(\Theta\)
  с вероятностью \(\gamma\).
\end{remark}

\begin{remark}
  Вероятность \(\alpha = 1 - \gamma\) называется уровнем значимости
  доверительного интервала.
\end{remark}

\begin{remark}
  Обычно пытаемся получить доверительный интервал симметричный относительно
  несмещенной точной оценки \(\Theta^*\), но не всегда это возможно.
\end{remark}

\begin{remark}
  Стандартные уровни надежности \(0.9\), \(0.95\), \(0.99\) и \(0.999\).
  Наиболее распространенный \(0.95\).
\end{remark}

\begin{lemma}
  Если случайная величина \(\xi\) имеет симметричное распределение относительно
  точки \(x = 0\), то \(\prob{\abs{\xi} < t} = 2 F(t) - 1\).
\end{lemma}

\begin{proof}
  Под симметричным распределением подразумевается распределение с симметричным
  графиком плотности. Тогда

  \begin{equation*}
    \prob{\abs{\xi} < t} 
    = 2 \prob{0 < \xi < t}
    = 2 \prh{F(t) - F(0)}
    = 2 F(t) - 1
  \end{equation*}

  т.к. \(F(0) = 0.5\) в силу симметрии.
\end{proof}

\subheader{Доверительные интервалы для параметров нормального распределения}

Пусть \(\sample{X} = \prh{X_1, \dotsc, X_n}\)~--- выборка объема \(n\) из
\(\ndist{a}{\sigma^2}\). Необходимо построить доверительные интервалы. Возможны
\(4\) ситуации.

\subsubheader{I.}{Доверительный интервал для параметра \(a\) при известном
значении параметра \(\sigma^2\)}

По \ref{lem:base-theorem-1} имеем

\begin{equation*}
  \sqrt{n} \cdot \frac{\avg{x} - a}{\sigma} \in \ndist{0}{1}
\end{equation*}

Будем искать интервал в виде

\begin{equation*}
  \prob{-t_{\gamma} < \sqrt{n} \cdot \frac{\avg{x} - a}{\sigma} < t_{\gamma}}
  = \prob{\abs{\sqrt{n} \cdot \frac{\avg{x} - a}{\sigma}} < t_{\gamma}} 
  = 2 \Phi \prh{t_{\gamma}}
  = \gamma
  \implies
  \Phi \prh{t_{\gamma}} = \frac{\gamma}{2}
\end{equation*}

Таким образом \(t_{\gamma}\)~--- обратное значение функции Лапласа или квантиль
уровня \(\frac{1 + \gamma}{2}\) стандартного нормального распределения. Остается
решить изначальное неравенство относительно параметра \(a\), получаем

\begin{equation*}
  \begin{aligned}
    -t_{\gamma} < \sqrt{n} \cdot \frac{\avg{x} - a}{\sigma} < t_{\gamma}
  \\
    -t_{\gamma} \frac{\sigma}{\sqrt{n}} < \avg{x} - a
      < t_{\gamma} \frac{\sigma}{\sqrt{n}}
  \\
    \avg{x} - t_{\gamma} \frac{\sigma}{\sqrt{n}} < a
      < \avg{x} + t_{\gamma} \frac{\sigma}{\sqrt{n}}
  \end{aligned}
\end{equation*}

Итак, получили доверительный интервал для параметра \(a\) надежности \(\gamma\)
вида \(\display{\interval{\avg{x} - t_{\gamma} \frac{\sigma}{\sqrt{n}}}
{\avg{x} + t_{\gamma} \frac{\sigma}{\sqrt{n}}}}\), где \(t_{\gamma}\) находится
из условия \(\display{\Phi \prh{t_{\gamma}} = \frac{\gamma}{2}}\).

\subsubheader{II.}{Доверительный интервал для параметра \(a\) при неизвестном
значении параметра \(\sigma^2\)}

По \ref{lem:base-theorem-4} имеем

\begin{equation*}
  \sqrt{n} \cdot \frac{\avg{x} - a}{S} \in \Tdist{n - 1}
\end{equation*}

Проведем аналогичные рассуждения, что и в прошлом пункте.

\begin{equation*}
  \prob{-t_{\gamma} < \sqrt{n} \cdot \frac{\avg{x} - a}{S} < t_{\gamma}}
  = \prob{\abs{\sqrt{n} \cdot \frac{\avg{x} - a}{S}} < t_{\gamma}} 
  = 2 F_{\Tdist{n - 1}} \prh{t_{\gamma}} - 1
  = \gamma
  \implies
  F_{\Tdist{n - 1}} \prh{t_{\gamma}} = \frac{1 + \gamma}{2}
\end{equation*}

Таким образом \(t_{\gamma}\) это квантиль распределения Стьюдента уровня
\(\frac{1 + \gamma}{2}\). Решаем неравенство и получаем

\begin{equation*}
  \avg{x} - t_{\gamma} \frac{S}{\sqrt{n}} < a
    < \avg{x} + t_{\gamma} \frac{S}{\sqrt{n}}
\end{equation*}

\begin{remark}
  Заметим, что в обоих случаях получили симметричные интервалы относительно
  точечной оценки \(\avg{x}\).
\end{remark}

\begin{remark}
  В Excel квантиль \(t_{\gamma}\) удобнее находить как \texttt{\(t_{\gamma} =\)
  СТЬЮДЕНТ.ОБ.2Х \((1 - \gamma, n - 1)\)}.
\end{remark}

\subsubheader{III.}{Доверительный интервал для параметра \(\sigma^2\) при
неизвестном значении параметра \(a\)}

По \ref{lem:base-theorem-3} имеем

\begin{equation*}
  \frac{(n - 1) S^2}{\sigma^2} \in \Hdist{n - 1}
\end{equation*}

Пусть \(\chi_1^2\) и \(\chi_2^2\) это квантили распределения \(\Hdist{n - 1}\)
уровней \(\frac{1 - \gamma}{2}\) и \(\frac{1 + \gamma}{2}\). Тогда

\begin{equation*}
  \prob{\chi_1^2 < \frac{(n - 1) S^2}{\sigma^2} < \chi_2^2}
  = F_{\Hdist{n - 1}} \prh{\chi_2^2} - F_{\Hdist{n - 1}} \prh{\chi_1^2}
  = \frac{1 + \gamma}{2} - \frac{1 - \gamma}{2}
  = \gamma
\end{equation*}

Решим неравенство относительно \(\sigma^2\).

\begin{equation*}
  \begin{aligned}
    \chi_1^2 < \frac{(n - 1) S^2}{\sigma^2} < \chi_2^2
  \\
    \frac{1}{\chi_2^2} < \frac{\sigma^2}{(n - 1) S^2} < \frac{1}{\chi_1^2}
  \\
    \frac{(n - 1) S^2}{\chi_2^2} < \sigma^2 < \frac{(n - 1) S^2}{\chi_1^2}
  \end{aligned}
\end{equation*}

Получили доверительный интервал \(\display{\interval{\frac{(n - 1) S^2}
{\chi_2^2}}{\frac{(n - 1) S^2}{\chi_1^2}}}\), где \(\chi_1^2\) и \(\chi_2^2\)
это квантили распределения \(\Hdist{n - 1}\) уровней \(\frac{1 - \gamma}{2}\) и
\(\frac{1 + \gamma}{2}\).

\begin{remark}
  Получили доверительный интервал для среднего квадратического отклонения
  \(\display{\interval{\frac{S \sqrt{n - 1}}{\chi_2}}{\frac{S \sqrt{n - 1}}
  {\chi_1}}}\).
\end{remark}

\subsubheader{IV.}{Доверительный интервал для параметра \(\sigma^2\) при
известном значении параметра \(a\)}

По \ref{lem:base-theorem-2} имеем

\begin{equation*}
  \sum_{i = 1}^n \prh{\frac{x_i - a}{\sigma}}^2
  = \frac{n \prh{\sigma^2}^*}{\sigma^2} \in \Hdist{n}
\end{equation*}

где \(\prh{\sigma^2}^* = \frac{1}{n} \sum_{i = 1}^n \prh{x_i - a}^2\). Пусть
\(\chi_1^2\) и \(\chi_2^2\) это квантили распределения \(\Hdist{n}\) уровней
\(\frac{1 - \gamma}{2}\) и \(\frac{1 + \gamma}{2}\). Тогда

\begin{equation*}
  \prob{\chi_1^2 < \frac{n \prh{\sigma^2}^*}{\sigma^2} < \chi_2^2}
  = F_{\Hdist{n}} \prh{\chi_2^2} - F_{\Hdist{n}} \prh{\chi_1^2}
  = \frac{1 + \gamma}{2} - \frac{1 - \gamma}{2}
  = \gamma
\end{equation*}

Аналогично третьему пункту решим неравенство и получим

\begin{equation*}
  \frac{n \prh{\sigma^2}^*}{\chi_2^2} < \sigma^2
    < \frac{n \prh{\sigma^2}^*}{\chi_1^2}
\end{equation*}

Итак, доверительный для \(\sigma^2\) имеет вид \(\display{\interval{\frac{n
\prh{\sigma^2}^*}{\chi_2^2}}{\frac{n \prh{\sigma^2}^*}{\chi_1^2}}}\), где
\(\chi_1^2\) и \(\chi_2^2\) это квантили распределения \(\Hdist{n}\) уровней
\(\frac{1 - \gamma}{2}\) и \(\frac{1 + \gamma}{2}\).

\begin{remark}
  Получили доверительный интервал для \(\sigma\) вида
  \(\display{\interval{\frac{\sqrt{n \prh{\sigma^2}^*}}{\chi_2}}{\frac{\sqrt{n
  \prh{\sigma^2}^*}}{\chi_1}}}\).
\end{remark}

\begin{remark}
  Доверительные интервалы полученные в третьем и четвертом пунктах не являются
  симметричными относительно \(S^2\), но есть формулы для симметричных
  доверительных интервалов.
\end{remark}

\begin{example}
  Пусть \(X \in \ndist{a}{\sigma^2}\) причем известно, что \(\sigma = 3\). В
  результате обработки выборки объема \(n = 36\) получили \(\avg{x} = 4.1\).
  Найти доверительный интервал для параметра \(a\) надежности \(\gamma = 0.95\).

  \solution{} Найдем \(t_{\gamma}\).

  \begin{equation*}
    2 \Phi \prh{t_{\gamma}} = 0.95
    \implies \Phi \prh{t_{\gamma}} = 0.475
    \implies t_{\gamma} = 1.96
  \end{equation*}

  Применим формулу из первого пункта.

  \begin{equation*}
    \begin{aligned}
      \avg{x} - t_{\gamma} \frac{S}{\sqrt{n}} & < a
      < & \avg{x} + t_{\gamma} \frac{S}{\sqrt{n}}
    \\
      4.1 - 1.96 \cdot \frac{3}{\sqrt{36}} & < a
        < & 4.1 + 1.96 \cdot \frac{3}{\sqrt{36}}
    \\
      3.12 & < a < & 5.08
    \end{aligned}
  \end{equation*}
\end{example}

\begin{example}
  Пусть \(X \in \ndist{a}{\sigma^2}\). В результате обработки выборки объема \(n
  = 25\) получили \(\avg{x} = 42.32\) и \(S = 6.4\). Найти доверительный
  интервал для параметра \(a\) при уровне надежности \(\gamma = 0.95\).

  \solution{} Найдем \(t_{\gamma}\). Получим \texttt{СТЬЮДЕНТ.ОБР.2Х \((1 -
  0.95, 25 - 1) = 2.064\)}. Подставим в формулу из второго пункта.

  \begin{equation*}
    \begin{aligned}
      \avg{x} - t_{\gamma} \frac{S}{\sqrt{n}} & < a
        < & \avg{x} + t_{\gamma} \frac{S}{\sqrt{n}}
    \\
      42.32 - 2.064 \cdot \frac{6.4}{\sqrt{25}} & < a
        < & 42.32 + 2.064 \cdot \frac{6.4}{\sqrt{25}}
    \\
      39.678 & < a < & 44.962
    \end{aligned}
  \end{equation*}
\end{example}
